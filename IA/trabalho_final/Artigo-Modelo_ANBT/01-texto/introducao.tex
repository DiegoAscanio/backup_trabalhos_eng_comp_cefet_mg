% -----------------------------------------------------------------------------
%   Arquivo: ./01-texto/introducao.tex
% -----------------------------------------------------------------------------



\section{Introdução}\label{sec:introducao}


%\lettrine[nindent=0em,lines=3]{A}		% Aumenta a primeira letra do primeiro parágrafo da seção

A localização de prestadores de serviço de confiança é uma tarefa que muitas vezes demanda uma grande quantidade de tempo, assim como uma boa dose de sorte para encontrar um prestador de qualidade. Nos dias atuais, as redes sociais promoveram uma interligação massiva entre pessoas de diversas partes do mundo e um professor do Centro Federal de Educação Tecnológica - Prof. Dr. Fábio Rocha da Silva viu um potencial de utilização das redes sociais para o encontro de prestadores de serviço, dando essa ideia para seus alunos em sala de aula.

Cursando a disciplina de Inteligência Artificial, o autor deste artigo viu uma técnica de busca em grafos - Busca em Profunidade Limitada (Depth Limited Search), baseada na busca em profundidade DFS, existente no livro de algoritmos de \citeonline{Cormen} que pôde ser utilizada para implementar a ideia proposta.

A implementação do trabalho utilizou a linguagem Python para simular a rede social e implementar a busca em profundidade limitada, assim como a biblioteca vis, desenvolvida em javascript para visualizar a rede social gerada.
