% -----------------------------------------------------------------------------
%   Arquivo: ./01-texto/conclusao.tex
% -----------------------------------------------------------------------------


\section{Conclusão}\label{sec:conclusao}

A implementação da busca em profundidade limitada mostrou-se ser adequada para a implementação da idéia proposta pelo professor Dr. Fábio Rocha da Silva.
Apesar dele não ter mencionado as restrições de contactar somente amigos ou amigos de amigos para prestação de serviços, os autores deste artigo acharam por bem implementar dessa forma, de maneira a tentar aproximar a realidade do ambiente virtual da maneira mais fidedigna possível no que diz a questão de confiança. É bem mais fácil confiar em amigos ou amigos de amigos do que em completos desconhecidos. Porém, com o grande avanço das redes sociais, hoje tornou-se comum adicionar pessoas que não possuem forte vinculo de amizade, pessoas que são apenas conhecidas. Por isso, na questão de confiança, essa parte pode ficar um pouco comprometida para usuários que apresentam a característica de ter amigos além do seu círculo de amizade íntimo cotidiano em suas redes sociais. Porém, esse trabalho abre um caminho para uma área não muito bem explorada até o momento e que pode apresentar grande potencial rentável.

As seguintes funcionalidades podem ser aprimoradas a este trabalho para uma maior agregação de valor:

- Distinguir conhecidos de amigos
- Implementar um sistema de classificação de prestadores de serviço, assim como de clientes
- Implementar um sistema de pagamentos entre prestadores e clientes que permita realizar seguro para prestação de serviços, bem como a fração de pagamentos em etapas concluídas do serviço
- Implementar um sistema que permita a associação em grupos, tanto por prestadores de serviço quanto por clientes, para facilitar ambas as partes
